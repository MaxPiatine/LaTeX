\documentclass[11pt]{article} 
%\usepackage{amsbsy} % for \boldsymbol and \pmb 
%\usepackage{graphicx} % To include pdf files!
\usepackage{amsmath}
\usepackage{amsbsy}
\usepackage{amsfonts}
\usepackage{enumerate}
\usepackage[colorlinks=true, pdfstartview=FitV, linkcolor=blue, citecolor=blue, urlcolor=blue]{hyperref} % For links
\usepackage{fullpage}
\pagestyle{empty}
\usepackage{pgf,pgfplots,tikz}
\usepackage{amsmath,amssymb,amsthm}
\usepackage{tikz}
\newcommand{\overrightharp}[1]{\hat{#1}}
\DeclareMathOperator{\proj}{proj}
\DeclareMathOperator{\years}{years}
\DeclareMathOperator{\cm}{cm}
\newcommand{\vct}{\mathbf}
\newcommand{\vctproj}[2][]{\proj_{{#1}}\vct{#2}}
\newtheorem{theorem}{Theorem}
\DeclareMathOperator{\m}{m}
\DeclareMathOperator{\kg}{kg}
\DeclareMathOperator{\N}{N}
\DeclareMathOperator{\Or}{or}
\DeclareMathOperator{\J}{J}
\DeclareMathOperator{\s}{s}
\DeclareMathOperator{\g}{g}
\DeclareMathOperator{\W}{W}
\DeclareMathOperator{\Heatoms}{He\hspace{1mm} atoms}
\DeclareMathOperator{\MeV}{MeV}
\DeclareMathOperator{\tr}{tr}
\DeclareMathOperator*{\E}{\mathbb{E}}
\newcommand{\norm}[1]{\left\lVert#1\right\rVert}
\usepackage{graphicx}
\newcommand{\abs}[1]{\lvert#1\rvert}
\DeclareMathOperator{\diverge}{div\,}
\DeclareMathOperator{\curl}{curl\,}
\title{\textbf{JCP321 POTW5}
\author{Maxim Piatine\\1005303100}}
\date{}
\DeclareMathOperator{\lineint}{\int \mathbf{v}\cdot d\mathbf{l}}
\DeclareMathOperator{\surfint}{\int \mathbf{v}\cdot d\mathbf{a}}
\begin{document}
\maketitle
\[\hat{T}=\frac{\hat{p}^2}{2m} \text{ , where $\hat{p}$ equals to } \hat{p}=i(\hat{a}_+-\hat{a}_-)\sqrt{\frac{\hbar m \omega}{2}}\]
\[\E[T]=\int^\infty_{-\infty}\psi^*\hat{T}\psi dx = 
\int^\infty_{-\infty}\psi^*\left(\frac{\hat{p}^2}{2m}\right)\psi dx=
\frac{1}{2m}\int^\infty_{-\infty}\psi^*\left(i(\hat{a}_+-\hat{a}_-)\sqrt{\frac{\hbar m \omega}{2}}\right)^2\psi dx\]
\[=\frac{1}{2m}\frac{-\hbar m \omega}{2}\int^\infty_{-\infty}\psi^*(\hat{a}_+-\hat{a}_-)^2\psi dx\]
\[(\hat{a}_+-\hat{a}_-)^2=(\hat{a}_+\hat{a}_+) - (\hat{a}_+\hat{a}_-) - (\hat{a}_- \hat{a}_+) + (\hat{a}_-\hat{a}_-)\]
From lecture and example 2.5 in the book, we can conclude that $\hat{a}_+\hat{a}_+ \psi$ and $\hat{a}_-\hat{a}_- \psi$ equate to zero due to orthogonality between $\psi_n$.
\vspace{5mm}
\\Properties used: 
\[(1)\hat{a}_+\hat{a}_-\psi = n\]
\[(2)\hat{a}_-\hat{a}_+\psi = n+1\]
\[(3)\int^\infty_{-\infty}\psi^*\psi dx= 1\]
\vspace{5mm}
\[=\frac{-\hbar \omega}{4}\int^\infty_{-\infty}\psi^*(-\hat{a}_+\hat{a}_-\psi-\hat{a}_-\hat{a}_+\psi) dx=
\frac{\hbar \omega}{4}\left( \int^\infty_{-\infty}(n)\psi^*\psi dx + \int^\infty_{-\infty}(n+1)\psi^*\psi dx\right)\]
\[=\frac{\hbar \omega}{4}(n+n+1)=\frac{\hbar \omega}{4}(2)(n+\frac{1}{2})=\frac{\hbar \omega}{2}(n+\frac{1}{2})=\E[T]\]
\end{document}
