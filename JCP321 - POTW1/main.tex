\documentclass[11pt]{article} 
%\usepackage{amsbsy} % for \boldsymbol and \pmb 
%\usepackage{graphicx} % To include pdf files!
\usepackage{amsmath}
\usepackage{amsbsy}
\usepackage{amsfonts}
\usepackage{enumerate}
\usepackage[colorlinks=true, pdfstartview=FitV, linkcolor=blue, citecolor=blue, urlcolor=blue]{hyperref} % For links
\usepackage{fullpage}
\pagestyle{empty}
\usepackage{pgf,pgfplots,tikz}
\usepackage{amsmath,amssymb,amsthm}
\usepackage{tikz}
\newcommand{\overrightharp}[1]{\hat{#1}}
\DeclareMathOperator{\proj}{proj}
\DeclareMathOperator{\years}{years}
\DeclareMathOperator{\cm}{cm}
\newcommand{\vct}{\mathbf}
\newcommand{\vctproj}[2][]{\proj_{{#1}}\vct{#2}}
\newtheorem{theorem}{Theorem}
\DeclareMathOperator{\m}{m}
\DeclareMathOperator{\kg}{kg}
\DeclareMathOperator{\N}{N}
\DeclareMathOperator{\Or}{or}
\DeclareMathOperator{\J}{J}
\DeclareMathOperator{\s}{s}
\DeclareMathOperator{\g}{g}
\DeclareMathOperator{\W}{W}
\DeclareMathOperator{\Heatoms}{He\hspace{1mm} atoms}
\DeclareMathOperator{\MeV}{MeV}
\DeclareMathOperator{\tr}{tr}
\newcommand{\norm}[1]{\left\lVert#1\right\rVert}
\usepackage{graphicx}
\newcommand{\abs}[1]{\lvert#1\rvert}
\DeclareMathOperator{\diverge}{div\,}
\DeclareMathOperator{\curl}{curl\,}
\title{\textbf{JCP321 POTW1}
\author{Maxim Piatine\\1005303100}}
\date{}
\DeclareMathOperator{\lineint}{\int \mathbf{v}\cdot d\mathbf{l}}
\DeclareMathOperator{\surfint}{\int \mathbf{v}\cdot d\mathbf{a}}
\begin{document}
\maketitle
\begin{center}
\begin{tikzpicture}
\node[circle,fill=black,inner sep=2.5mm] (a) at (1,0);

\draw[decoration={aspect=0.3, segment length=3mm, amplitude=3mm,coil},decorate] (1,5) -- (a); 

\fill [pattern = north east lines] (-1,5) rectangle (3,5.2);
\draw[thick] (-1,5) -- (3,5);

% Weight Force
\draw[-latex][blue](0,0) -- node[right]{$\vec{g}$}++(0,-1) ;

% Tension Force
\draw[-latex](0,0) -- node[right]{$\vec{k}$}++(0,1)

\end{tikzpicture}
\end{center}
\vspace{5mm}
Starting with potential function $V(q)$, assuming that the mass is centered ($q=0$):
\[V_1(q_1)=\frac{1}{2}kq_1^2\] 
\[V_2(q_1)=mgq_1\]
\[V(q_1)=\frac{1}{2}kq_1^2+mgq_1\]
Kinetic energy from the spring:
\[T=\frac{1}{2}m\dot{q}_1^2\]
Hamiltonian:
\[H=T+V=\frac{1}{2}m\dot{q}_1^2+\frac{1}{2}kq_1^2+mgq_1\]
However, we need to represent it in momentum and position ($H(p,q)$):
\[T=\frac{p_1^2}{2m}=\frac{1}{2}m\dot{q}^2_1\]
\[H=\frac{p_1^2}{2m}+\frac{1}{2}kq_1^2+mgq_1\]
\newpage
Partial derivatives of the Hamiltonian function:
\[\frac{\partial H}{\partial p_1}=\frac{p_1}{m}=\dot{q}_1 \rightarrow p_1=m\dot{q}_1\]
\[\frac{\partial H}{\partial q_1}=kq_1+mg=-\dot{p}_1\]
Now to find the equation of motion:
\[kq_1+mg=-(m\Ddot{q}_1)\]
\[m\Ddot{q}_1+kq_1+mg=0\]
\[\Ddot{q}_1+\frac{kq_1}{m}+g=0\]
\end{document}
