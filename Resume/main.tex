\documentclass{article}
\usepackage[cm]{fullpage}
\usepackage{color}
\usepackage{hyperref}

\hypersetup{breaklinks=true,%
pagecolor=white,%
colorlinks=true,%
linkcolor=cyan,%
urlcolor=MyDarkBlue}

\definecolor{MyDarkBlue}{rgb}{0,0.0,0.45}

%%%%%%%%%%%%%%%%%%%%%%%%%%
% Formatting parameters  %
%%%%%%%%%%%%%%%%%%%%%%%%%%

\newlength{\tabin}
\setlength{\tabin}{1em}
\newlength{\secsep}
\setlength{\secsep}{0.1cm}

\setlength{\parindent}{0in}
\setlength{\parskip}{0in}
\setlength{\itemsep}{0in}
\setlength{\topsep}{0in}
\setlength{\tabcolsep}{0in}

\definecolor{contactgray}{gray}{0.3}
\pagestyle{empty}

%%%%%%%%%%%%%%%%%%%%%%%%%%
%  Template Definitions  %
%%%%%%%%%%%%%%%%%%%%%%%%%%

\newcommand{\lineunder}{\vspace*{-8pt} \\ \hspace*{-6pt} \hrulefill \\ \vspace*{-15pt}}
\newcommand{\name}[1]{\begin{center}\textsc{\Huge#1}\\\end{center}}
\newcommand{\program}[1]{\begin{center}\textsc{#1}\end{center}}
\newcommand{\contact}[1]{\begin{center}\color{contactgray}{\small#1}\end{center}}

\newenvironment{tabbedsection}[1]{
  \begin{list}{}{
      \setlength{\itemsep}{0pt}
      \setlength{\labelsep}{0pt}
      \setlength{\labelwidth}{0pt}
      \setlength{\leftmargin}{\tabin}
      \setlength{\rightmargin}{\tabin}
      \setlength{\listparindent}{0pt}
      \setlength{\parsep}{0pt}
      \setlength{\parskip}{0pt}
      \setlength{\partopsep}{0pt}
      \setlength{\topsep}{#1}
    }
  \item[]
}{\end{list}}

\newenvironment{nospacetabbing}{
    \begin{tabbing}
}{\end{tabbing}\vspace{-1.2em}}

\newenvironment{resume_header}{}{\vspace{0pt}}


\newenvironment{resume_section}[1]{
  \filbreak
  \vspace{2\secsep}
  \textsc{\large#1}
  \lineunder
  \begin{tabbedsection}{\secsep}
}{\end{tabbedsection}}

\newenvironment{resume_subsection}[2][]{
  \textbf{#2} \hfill {\footnotesize #1} \hspace{2em}
  \begin{tabbedsection}{0.5\secsep}
}{\end{tabbedsection}}

\newenvironment{subitems}{
  \renewcommand{\labelitemi}{-}
  \begin{itemize}
      \setlength{\labelsep}{1em}
}{\end{itemize}}

\newenvironment{resume_employer}[4]{
  \vspace{\secsep}
  \textbf{#1} \\ 
  \indent {\small #2} \hfill {\footnotesize#3 (#4)}
  \begin{tabbedsection}{0pt}
  \begin{subitems}
}{\end{subitems}\end{tabbedsection}}


%%%%%%%%%%%%%%%%%%%%%%%%%%
%     Start Document     %
%%%%%%%%%%%%%%%%%%%%%%%%%%

\begin{document}

\begin{resume_header}
\name{Maxim Piatine}
\contact{max.piatine@hotmail.com \hspace{2cm} 438-830-8284 \hspace{2cm} \url{https://github.com/maxpiatine}}
\end{resume_header}

\begin{resume_section}{About Me}
  \begin{nospacetabbing}

  \textbf{Technical Skills}  \= C/C++, Python, Excel VBA, MATLAB, HTML, CSS, JavaScript, R, Git, \LaTeX\\*
  \textbf{Personal Skills}  \> Analytical, Creative, Communicating, Collaborating, Organization, Public Speaking\\*
  \textbf{Languages} \> Fluent in French and English; Conversational Proficiency in Russian\\*
  \textbf{Art Technology} \> Adobe Photoshop, Adobe Premiere Pro, Illustrator\\*
  \textbf{Interests} \> Athletics, Quantum Computing, Puzzles, Learning, Scripting, Chess, Languages\\*
  \end{nospacetabbing}

\end{resume_section}

\begin{resume_section}{Education}
  \begin{resume_subsection}[Toronto, ON (2018 -- 2022)]{University of Toronto}
   \begin{subitems}
      \item Honours Bachelor of Science (ER HBSC)
    \item Physics and Applied Statistics Major, Mathematical Sciences Minor
    \item Cumulative Average of $80\%$ (3.5/4.0) 
    \begin{subitems}
      \item In the last two years of university
    \end{subitems}
    \item Courses
    \begin{subitems}
    \item Statistical Methods of Machine Learning, Stochastic Processes, Statistical Mechanics,\\ Computation Modeling in Physics, Computer Science, Quantum Mechanics, Classical Electrodynamics
    \end{subitems}
   \end{subitems}
  \end{resume_subsection}
\end{resume_section}

\begin{resume_section}{Work Experience}
  \begin{resume_subsection}[Ottawa, ON(October 2017 -- April 2019)]{Middle Manager, Roast Cafe}
  \begin{subitems}
  \item Led the company's financial, budgeting, and accounting via Excel. Programmed automatic Pivot Tables with Macros that displays the company's weekly/monthly net income.
  \item Front-end developed the website using HTML, CSS, and JavaScript to display the Menu, About Us, and different Social Medias linked to Roast Cafe.
  \item Directed and marketed the company's social media account via Instagram and Facebook.
  \end{subitems}
      
  \end{resume_subsection}
\end{resume_section}

\begin{resume_section}{Competitions and Personal Projects}
  \begin{resume_subsection}[(December 2021)]{Statistical Methods of Machine Learning Kaggle Competition}
  \begin{subitems}
    \item  A $10\%$ R assignment assigned to every student in the class. The top 10 students get bonus marks.
    \item Developed a program that recursively goes through $x$ number of variables and outputs the lowest root mean square error by Pruning trees. A Regression Trees method we learned in class.
    \item Placed in $13^{\text{th}}$ place out of 136 students.
    \end{subitems}
 \end{resume_subsection}


  \begin{resume_subsection}[(April 2021)]{Credit Card Validation}
  \begin{subitems}
    \item Wrote Luhn's Algorithm in C that allows to check the validity of someone's credit card.
    \item Currently implementing an automated working service that allows clients to purchase online.
    \end{subitems}
  \end{resume_subsection}



  \begin{resume_subsection}[(December 2020 - January 2021)]{Website Portfolio}
    \begin{subitems}
        \item Made a Portfolio website displaying my skills with HTML, CSS, and JavaScript.
        \item Part of my Portfolio website had a Grade Calculator, that enabled me to calculate my grades in a class and measure my academic progress throughout the years. Used by $15+$ people.
        \item Also, implemented a Readability feature that allowed me to post my advanced Physics lab reports and assignments. In return I'd get a readability score ranging from 0 to 100.
    \end{subitems}
  \end{resume_subsection}
  
  \begin{resume_subsection}[(May 2021)]{Filter}
    \begin{subitems}
    \item Developed a program that enables me to input a digital picture, and outputs a variety of amount of filtered pictures that I want using C.
    \item The program reads through every pixel and mathematically changes the RGB depending on the filter wanted.
    \end{subitems}
  \end{resume_subsection}
  
    \begin{resume_subsection}[(February 2022)]{NFT Generator}
        \begin{subitems}
        \item Wrote an algorithm in Python to generates $10,000+$ different Non Fungible Tokens in under 5 minutes.
        \item Using win32com client module to open Adobe Photoshop and access a variety of files to make layers visible and invisible simultaneously. Then outputting all the different varieties of NFTs depending on the filters and seed variation.
        \item Currently, working on diminishing the run time complexity using vectorization.
        \end{subitems}
    \end{resume_subsection}


\end{resume_section}
\end{document}