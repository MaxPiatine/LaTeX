\documentclass[11pt]{article} 
%\usepackage{amsbsy} % for \boldsymbol and \pmb 
%\usepackage{graphicx} % To include pdf files!
\usepackage{amsmath}
\usepackage{amsbsy}
\usepackage{amsfonts}
\usepackage{enumerate}
\usepackage[colorlinks=true, pdfstartview=FitV, linkcolor=blue, citecolor=blue, urlcolor=blue]{hyperref} % For links
\usepackage{fullpage}
\pagestyle{empty}
\usepackage{pgf,pgfplots,tikz}
\usepackage{amsmath,amssymb,amsthm}
\usepackage{tikz}
\newcommand{\overrightharp}[1]{\hat{#1}}
\DeclareMathOperator{\proj}{proj}
\DeclareMathOperator{\years}{years}
\DeclareMathOperator{\cm}{cm}
\newcommand{\vct}{\mathbf}
\newcommand{\vctproj}[2][]{\proj_{{#1}}\vct{#2}}
\newtheorem{theorem}{Theorem}
\DeclareMathOperator{\m}{m}
\DeclareMathOperator{\kg}{kg}
\DeclareMathOperator{\N}{N}
\DeclareMathOperator{\Or}{or}
\DeclareMathOperator{\J}{J}
\DeclareMathOperator{\s}{s}
\DeclareMathOperator{\g}{g}
\DeclareMathOperator{\W}{W}
\DeclareMathOperator{\Heatoms}{He\hspace{1mm} atoms}
\DeclareMathOperator{\MeV}{MeV}
\DeclareMathOperator{\tr}{tr}
\DeclareMathOperator*{\E}{\mathbb{E}}
\newcommand{\norm}[1]{\left\lVert#1\right\rVert}
\usepackage{graphicx}
\newcommand{\abs}[1]{\lvert#1\rvert}
\DeclareMathOperator{\diverge}{div\,}
\DeclareMathOperator{\curl}{curl\,}
\title{\textbf{JCP321 POTW7}
\author{Maxim Piatine\\1005303100}}
\date{}
\DeclareMathOperator{\lineint}{\int \mathbf{v}\cdot d\mathbf{l}}
\DeclareMathOperator{\surfint}{\int \mathbf{v}\cdot d\mathbf{a}}
\begin{document}
\maketitle
\section*{(a)}
\[\hat{L}_x=-i\hbar\left(
-\sin{\phi}\frac{\partial}{\partial \theta}-\cot{\theta}\cos{\phi}\frac{\partial}{\partial \phi}
\right)\]
\[l=1\text{ },\text{ }m_l=-1 \text{ } \Rightarrow \text{ } Y^{m_l}_l=Y^{-1}_{1}=\left(\frac{3}{8\pi}\right)^{\frac{1}{2}}\sin{\theta}e^{-i\phi}\]
\[\hat{L}_xY^{-1}_1=-i\hbar\left(
-\sin{\phi}\frac{\partial}{\partial \theta}\left(\left(\frac{3}{8\pi}\right)^{\frac{1}{2}}\sin{\theta}e^{-i\phi}\right)-\cot{\theta}\cos{\phi}\frac{\partial}{\partial \phi}
\left(\left(\frac{3}{8\pi}\right)^{\frac{1}{2}}\sin{\theta}e^{-i\phi}\right)\right)\]
\[=-i\hbar\left(-\left(\frac{3}{8\pi}\right)^\frac{1}{2}\sin{\phi}e^{-i\phi}\frac{\partial (\sin{\theta})}{\partial \theta}-\left(\frac{3}{8\pi}\right)^{\frac{1}{2}}\cos{\theta}\cos{\phi}\frac{\partial (e^{-i\phi})}{\partial \phi}\right)\]
\[=\left(\frac{3}{8\pi}\right)^{\frac{1}{2}}\hbar\left(
i\sin{\phi}e^{-i\phi}\cos{\theta}+i\cos{\theta}\cos{\phi}(-i)e^{-i\phi}
\right)\]
\[=\left(\frac{3}{8\pi}\right)^{\frac{1}{2}}\hbar\cos{\theta} e^{-i\phi}\left(
i\sin{\phi}-i^2\cos{\phi}
\right)=\left(\frac{3}{8\pi}\right)^{\frac{1}{2}}\hbar\cos{\theta} e^{-i\phi}\left(
\cos{\phi}+i\sin{\phi}
\right)\]
\[=\left(\frac{3}{8\pi}\right)^{\frac{1}{2}}\hbar\cos{\theta} e^{-i\phi}e^{i\phi}=\left(\frac{3}{8\pi}\right)^{\frac{1}{2}}\hbar\cos{\theta}\]
\section*{(b)}
No, after applying the operator the answer does not correspond to $\alpha Y^{-1}_1$ where $\alpha$ is a constant! In fact, there is a whole cos term instead of sin and the exponent of phi that got cancelled out due to the Euler substitution. Thus, the stationary state is not an eigen function. 
\[\left(\frac{3}{8\pi}\right)^{\frac{1}{2}}\hbar\cos{\theta}\neq \alpha\left( \left(\frac{3}{8\pi}\right)^{\frac{1}{2}}\sin{\theta}e^{-i\phi}\right)\]

\section*{(c)}
\[\E\left[\hat{L}_x\right]=\int_0^{2\pi}\int^\pi_0Y^{-1*}_{\theta,\phi}\hat{L}_xY^{-1}_{\theta,\phi}\sin\theta d\theta d\phi=
\int_0^{2\pi}\int^\pi_0\left( \left(\frac{3}{8\pi}\right)^{\frac{1}{2}}\sin{\theta}e^{i\phi}\right)\left(\left(\frac{3}{8\pi}\right)^{\frac{1}{2}}\hbar\cos{\theta}\right)\sin\theta d\theta d\phi\]
\end{document}
